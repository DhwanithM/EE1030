%iffalse
\let\negmedspace\undefined
\let\negthickspace\undefined
\documentclass[journal,12pt,twocolumn]{IEEEtran}
\usepackage{cite}
\usepackage{amsmath,amssymb,amsfonts,amsthm}
\usepackage{algorithmic}
\usepackage{graphicx}
\usepackage{textcomp}
\usepackage{xcolor}
\usepackage{txfonts}
\usepackage{listings}
\usepackage{enumitem}
\usepackage{mathtools}
\usepackage{gensymb}
\usepackage{comment}
\usepackage[breaklinks=true]{hyperref}
\usepackage{tkz-euclide} 
\usepackage{listings}
\usepackage{gvv} 


%\def\inputGnumericTable{}                                 
\usepackage[latin1]{inputenc}                                
\usepackage{color}                                            
\usepackage{array}                                            
\usepackage{longtable}                                       
\usepackage{calc}                                             
\usepackage{multirow}                                         
\usepackage{hhline}                                           
\usepackage{ifthen}                                           
\usepackage{lscape}
\usepackage{tabularx}
\usepackage{array}
\usepackage{float}
\usepackage{multicol}

\newtheorem{theorem}{Theorem}[section]
\newtheorem{problem}{Problem}
\newtheorem{proposition}{Proposition}[section]
\newtheorem{lemma}{Lemma}[section]
\newtheorem{corollary}[theorem]{Corollary}
\newtheorem{example}{Example}[section]
\newtheorem{definition}[problem]{Definition}
\newcommand{\BEQA}{\begin{eqnarray}}
\newcommand{\EEQA}{\end{eqnarray}}
\newcommand{\define}{\stackrel{\triangle}{=}}
\theoremstyle{remark}
\newtheorem{rem}{Remark}

% Marks the beginning of the document
\begin{document}
\bibliographystyle{IEEEtran}
\vspace{3cm}

\title{Chapter 6 Sequences and Series}
\author{EE24BTECH11016- DHWANITH M}
\maketitle
\newpage
\bigskip

\renewcommand{\thefigure}{\theenumi}
\renewcommand{\thetable}{\theenumi}


\begin{enumerate}
    \setcounter{enumi}{5}
    \item If N is a natural number such that
\\ 
$n= p_{1}^{a_1}.p_{2}^{a_2}.p_{3}^{a_3}......p_{k}^{a_k} $ and $ p_{1},p_{2},....,p_{k} $ are distinct primes, then show that $ ln n \geq k ln2 $                              
     
		\hfill \brak{1985-5 Marks}              


	\item Find the sum of the series : \\          $\sum_{r=0}^{n} \brak{-1}^r \binom{n}{r}\sbrak{ \frac{1}{2^{r}}+\frac{3^{r}}{2^{2r}}+\frac{7^{r}}{2^{3r}}+\frac{15^{r}}{2^{4r}}.....$ up to $m$ terms$,] }$  
	    
	    \hfill \brak{1985-5 Marks }
	    
     \item Solve for $x$ the following equation:     
     
	     \hfill \brak{1987-3 Marks}          \\              
		     $   \log_{\brak{2x+3}}{\brak{6x^{2}+23x+21}}=4-\log_{\brak{3x+7}}{\brak{4x^{2}+12x+9}} $

	     \item If $ \log_{3}{2},\log_{3}{2^{x}-3} $ and $ \log_{3}{\brak{2^{x}-\frac{7}{2}}} $ are in arithmetic progression. Determine the value of $x$.  
     
	      \hfill \brak{1990 -4 Marks}
      

      \item Let $p$ be the first of $n$ arithmetic means between two numbers and $q$ the first of $n$ harmonic means between the same numbers . Show that $q$ does not lie between $p$ and $\brak{\frac{n+1}{n-1}}^{2}p$ 
       
	      \hfill \brak{1991 -4 Marks}
      

      \item  If $ S_1,S_2,S_3......,S_n $ are the sums of infinite geometric series whose first terms are $1,2,3,.....,n$ and whose common ratios are $ \frac{1}{2},\frac{1}{3},\frac{1}{4},.......,\frac{1}{n+1} $ respectively, then find the values of $ S_{1}^{2}+S_{2}^{2}+S_{3}^{2}+.....+S_{2n-1}^{2} $
      
	      \hfill \brak{1991 -4 Marks}
       

		\item  The real numbers $ x_{1},x_{2},x_{3} $ satisfying the equation $ x^{3}-x^{2}+\beta x+\gamma=0 $ are in AP. Find the intervals in which $ \beta $ and $\gamma$ lie.
       
			\hfill \brak{1996 -3 Marks}
      

      \item  Let $ a,b,c,d $ be real numbers in GP. If $u,v,w$ satisfy the system of equations  
    
	      \hfill \brak{1999 -10 Marks}
      
      $    u+2v+3w=6 $ \\
      $    4u+5v+6w=12 $ \\
      $    6u+9v=4 $ \\
      then show that the roots of the equation \\
		$\brak{\frac{1}{u}+\frac{1}{v}+\frac{1}{w}}x^{2}+\sbrak{\brak{b-c}^{2}+\brak{c-a}^{2}+\brak{d-b}^{2}}x+u+v+w=0 $ and $ 20x^{2}+10\brak{a-d}^{2}x-9=0 $ are reciprocals of each other.

      \item The fourth power of the common difference of an arithmetic progression with integer entries is added to the product of any four consecutive terms of it.Prove that the resulting sum is the square of an integer.
      
	      \hfill \brak{2000 -4 Marks}
    

      \item Let $ a_{1},a_{2},...,a_{n} $ be positive real numbers in geometric progression. For each $n$,let $ A_{n},G_{n},H_{n} $ be respectively, the arithmetic mean, geometric mean ,and harmonic mean of $ a_{1},a_{2},...,a_{n}.$ Find an expression for the geometric mean of $ G_{1},G_{2},...,G_{n} $ in terms of $ A_{1},A_{2},...,A_{n},H_{1},H_{2},...,H_{n}.$ 
      
	      \hfill \brak{2001 -5 Marks}                              
       
       \item Let $a,b$ be positive real numbers.If $ a,A_{1},A_{2},b $ are in arithmetic progression, $ a,G_{1},G_{2},b $ are in geometric progression and $ a,H_{1},H_{2},b $ are in harmonic progression, show that \\ 
	       $ \frac{G_{1}G_{2}}{H_{1}H_{2}}=\frac{A_{1}+A_{2}}{H_{1}+H_{2}}=\frac{\brak{2a+b}\brak{a+2b}}{9ab} $ 

		\hfill \brak{2002 -4 Marks}                             
     
	\item If $a,b,c$ are in A.P.,$a^{2},b^{2},c^{2} $ are in H.P., then prove that either $ a=b=c $ or $a,b,-\frac{c}{2}$ form a G.P. 
		                    
		\hfill \brak{2003 -4 Marks}                            

	\item If $ a_n=\frac{3}{4}-\brak{\frac{3}{4}}^{2}+\brak{\frac{3}{4}}^{3}+....\brak{-1}^{n-1}\brak{\frac{3}{4}}^{n} $ and $ b_{n}=1-a_{n}, $ then find the least natural number $n_{0}$ such that $ b_{n} \textgreater a_{n} \forall n \geq n_{0}. $ 
		             
		\hfill \brak{2006 -6 Marks}                             
 \end{enumerate} 


    \begin{center}    
      \textbf{PASSAGE-1}
    \end{center}
 Let $ V_{r} $ denote the sum of first $r$ terms of an arithmetic progression \brak{A.P} whose first term is $r$ and the common difference is $\brak{2r-1}$.Let $ T_{r}=V_{r+1}-V_{r}-2 $ and $ Q_{r}=T_{r+1}-T_{r}$ for $r=1,2,...$
 \\ 
 \begin{enumerate}

 \item The sum  $  V_{1}+V_{2}+...+V_{n} $  is 
 
	 \hfill \brak{2007 -4 Marks}                            
     \begin{enumerate}
         
	     \item $\frac{1}{12}n\brak{n+1}\brak{3n^{2}-n+1}$
	     \item $\frac{1}{12}n\brak{n+1}\brak{3n^{2}+n+2}$
	     \item $\frac{1}{2}n\brak{2n^{2}-n+1}$
	     \item $\frac{1}{3}\brak{2n^{3}-2n+3}$
    \end{enumerate} 

  \item $T_{r}$ is always 
                           
	  \hfill \brak{2007 -4 Marks}                    
                  \begin{multicols}{2}      
                        \begin{enumerate} 
	
       \item an odd number 
       \item an even number
	\item a prime number 
        \item composite number

	  \end{enumerate}
   \end{multicols}
    \item Which one of the following is a correct statement? 
          
	    \hfill \brak{2007 -4 Marks}                                  
     \begin{enumerate}
	\item $Q_{1},Q_{2},Q_{3},...$ are in A.P. with common difference 5 
	\item $Q_{1},Q_{2},Q_{3},...$ are in A.P. with common difference 6
	\item $Q_{1},Q_{2},Q_{3},...$ are in A.P. with common difference 11
	\item $Q_{1}=Q_{2}=Q_{3}=...$
     \end{enumerate}



 \end{enumerate}

	






    
\end{document}

